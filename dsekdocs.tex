\documentclass[a4paper, oneside]{ltxdoc}
\usepackage[
  colorlinks=true,
  bookmarks=true,
  bookmarksnumbered=true,
  bookmarksopen=true,
  bookmarksopenlevel=3,
  pdfauthor={D-sektionen},
  pdftitle={dsekdocs manual},
  pdfsubject={Manual for the dsekdocs LaTeX package},
]{hyperref}
\usepackage{dsek}
\usepackage{array}
\setlength{\parindent}{0mm}
\setlength{\parskip}{3mm}

\author{D-sektionen}
\title{The \textsf{dsekdocs} manual}

\begin{document}
\maketitle

\section{Overview}
This is the manual for version 0.1.1 of the \textsf{dsekdocs} packages for the
typesetting of documents for the D-guild at the technical faculty of Lund
university.

\section{The \textsf{dsek} package}
This package contains macros for the inclusion of D-guild related stuff in any
document.  To use it, write \cs{usepackage\{dsek\}}.

\subsection{Fonts}
The \textsf{dsek} package uses the free Domitian and \TeX Gyre Heros as stand
ins for non-free Palatino and Helvetica fonts.

It is worth noting that it by default uses old style figures, i.e. ``lower case
numbers'', by default.  If you want to turn that off somewhere, like maybe in a
table, use

\cs{addfontfeatures\{Numbers=\{Monospaced,Lining\}\}}

This makes the numbers full height and monospaced to increase readability.

\subsection{Commands}

\subsubsection{Text snippets}

These are commands that insert a snippet of text when used.  They exist to
prevent misspellings when writing hard to spell phrases.

\begin{center}
  \begin{tabular}{r | l}
    \cs{Dseklongname}  & \Dseklongname  \\
    \cs{idet}          & \idet          \\
    \cs{medaljk}       & \medaljk       \\
    \cs{overphos}      & \overphos      \\
    \cs{overpeppare}   & \overpeppare   \\
    \cs{medaljkmedlem} & \medaljkmedlem \\
  \end{tabular}
\end{center}

\subsubsection{Graphics}

These are commands that insert some kind of graphic when used.  All graphics
accept these options as the optional arguments (comma separated if multiple):

\begin{description}
  \item[\texttt{bw}] Forces the usage of the black and white version of the
    graphic.
  \item[\texttt{color}] Forces the usage of the colored version of the graphic
    (if it exists).
  \item[\texttt{height=}\textit{dim}] Sets the height of the graphic.
\end{description}

\begin{center}
  \begin{tabular}{r | c c}
    Command & Colored & Black and white \\ \hline
    \cs{Dsymbol} & N/A & \Dsymbol[bw, height=10mm] \\
    \cs{Dseksigil} & \Dseksigil[color, height=20mm] & \Dseksigil[bw,
      height=20mm] \\
    \cs{Cprogsigil} & \Cprogsigil[color, height=20mm] & \Cprogsigil[bw,
      height=20mm] \\
    \cs{Dprogsigil} & \Dprogsigil[color, height=20mm] & \Dprogsigil[bw,
      height=20mm] \\
  \end{tabular}
\end{center}

\subsubsection{Lists}
As it is often desirable to create so called ``att-listor'' to reflect decisions
taken or the proposed actions in motions, facilities to create them are provided:

\begin{center}
  \begin{tabular}{l}
    \cs{begin\{attlist\}}\\
    ... \\
    \cs{end\{attlist\}}
  \end{tabular}
\end{center}

For example:

\begin{center}
  \begin{tabular}{l | l}
    \begin{minipage}{0.5\linewidth}
      \texttt{Mötet beslutade} \\
      \cs{begin\{attlist\}}\\
      \cs{item} \texttt{steka pannkakor,} \\
      \cs{item} \texttt{D-sektionen är bäst!} \\
      \cs{end\{attlist\}}
    \end{minipage}
    &
    \begin{minipage}{0.5\linewidth}
      Mötet beslutade
      \begin{attlist}
      \item steka pannkakor,
      \item D-sektionen är bäst!
      \end{attlist}
    \end{minipage}
  \end{tabular}
\end{center}

If you would like to create a list in the same style but with another word in
place of \textit{att}, use the \texttt{boldlist} environment:

\begin{center}
  \begin{tabular}{l}
    \cs{begin\{boldlist\}}\marg{bullet word}\\
    ... \\
    \cs{end\{boldlist\}}
  \end{tabular}
\end{center}

\subsubsection{Signatures}
Finally, the \textsf{dsek} package also includes facilities for adding
signatures.

\begin{center}
  \cs{signature}\oarg{options}\marg{date and location}\marg{name}\marg{title}
\end{center}

For instance, a signature declaration that looks like this:

\begin{center}
  \cs{signature}\texttt{\{Lund, dag som ovan\}\{Trula Trulsson\}\{Ordförande\}}
\end{center}

results in

\begin{center}
  \signature{Lund, dag som ovan}{Trula Trulsson}{Ordförande}
\end{center}

\noindent
The signature command accepts the following options:
\begin{description}
  \item[\texttt{width=}\textlangle\textit{dim}\textrangle] Sets the width of the
    signature.  By default, it is set to the maximum of the widths of the
    entered texts.
  \item[\texttt{hspace=}\textlangle\textit{dim}\textrangle] Sets the height of
    the signature space.  By default it is set to 15mm.
\end{description}

\noindent
Additionally, the default signature height space for all signatures can be set
through this command:
\begin{center}
  \cs{setsignspaceheight}\marg{dim}
\end{center}

\subsection{Package options}
The \textsf{dsek} package accepts the following options, (comma separated if
multiple):

\begin{description}
  \item[\texttt{bw}] Makes graphics render in black and white by default.
  \item[\texttt{color}] Makes graphics render in color by default.
\end{description}

\section{The \textsf{dsekdoc} document class}
The \textsf{dsekdoc} document class is meant to be used for any guild related
documents for which the standard style is desired.  It is based of the
\textsf{article} document class, but changes fonts, paragraph styles, headers
and footers amongst a few other things.

\subsection{Options}
The \textsf{dsekdoc} document declares two special options:
\begin{description}
\item[\texttt{maketitle}] makes \textsf{dsekdoc} not redefine the \cs{maketitle}
  command (use the \cs{coverpage} command instead).
\item[\texttt{english}] tells the language package \textsf{polyglossia} that
  the main language for the document is English instead of Swedish.
\end{description}
All other arguments are passed through to the \textsf{article} document class.

\subsection{Commands}
Because of how the \cs{title}, \cs{author} and \cs{date} commands work, we can't
easily use what is given to them for our own purposes, so as a replacement we
have
\begin{center}
  \cs{settitle}\marg{title}, \cs{setauthor}\marg{author} and \cs{setdate}\marg{date}
\end{center}

These commands wrap around the usual \cs{title} commands etc., so there's no
need to use both.

\subsubsection{Headers}
In addition to the name of the guild and the date, the headers of the document
will include a short version of the document title and the meeting it is meant
for or when it was established.  Those are set with

\begin{center}
  \cs{setshorttitle}\marg{short title} and \cs{setmeeting}\marg{meeting}
\end{center}

\subsubsection{Cover page}
For longer, more important documents it is nice to have a cover page for the
document.  By default, the cover page is inserted with

\begin{center}
  \cs{maketitle} or \cs{coverpage}
\end{center}

In addition to the guild's name, the guild's sigil and the title of the
document, the cover page also features a small text that might be used as a
subtitle.  It can be set with

\begin{center}
  \cs{setcovptext}\marg{text}
\end{center}

\subsubsection{Reusing}
To use any of the properties we have described in this section, change the
``set'' in the command to ``use''.  So, to use the title you set with the
\cs{settitle} command in your document, you write \cs{usetitle} and it will be
inserted.

\section{The \textsf{dsekparagraphed} document class}
The \textsf{dsekparagraphed} document class is meant to be used for the creation
of guild documents that use a paragraphed structure, e.g. regulatory documents
and minutes.  The only user facing thing that it provides is the
\textsf{parasection} list environment.  It is intended to be the body of each
section of a regulatory document; it begins right after the \cs{section} call
and ends right before the next one.  Inside it, each paragraph is declared like
so:

\begin{center}
  \cs{paraitem}\marg{title}.
\end{center}

By default, all paragraphs follow the numbering format §\(X.Y\), where \(X\) is
the section number and \(Y\) is the paragraph number.  This format can be
altered with the following command:

\begin{center}
  \cs{setparanumformat}\marg{format}
\end{center}

The default format can be achieved like this:

\begin{center}
  \cs{setparanumformat}\texttt{\{§\cs{arabic}\{section\}.\cs{arabic}\{DsekParagraph\}\}}
\end{center}

If you want a single \texttt{paraitem} to have a special numbering (maybe not a
number at all), you can do so like this:

\begin{center}
  \cs{paraitem}\oarg{numbering}\marg{title}
\end{center}

This won't step the paragraph counter.

All paragraphs can be linked to using the \cs{hyperlink} command to create
clickable links in the document.  So, to link to the paragraph §1.2, you would
write

\begin{center}
  \cs{hyperlink}\texttt{\{par:§1.2\}\{See §1.2\} for further examples.}
\end{center}

\section{Technical notes}

\subsection{Compatability}

The \textsf{dsekdoc} and \textsf{dsekparagraphed} document classes make use of
the \href{https://ctan.org/pkg/fontspec}{\textsf{fontspec}} and
\href{https://ctan.org/pkg/polyglossia}{\textsf{polyglossia}} packages, both of
which are incompatible with \textsf{pdflatex}.  So, to use them, use
\textsf{xelatex} or \textsf{lualatex} instead.

\end{document}
