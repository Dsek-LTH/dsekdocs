\documentclass{dsekregulations}

\setadoptedon{HTMX 20XX}
\setrevisedon{VTMX 20XY}

\begin{document}
\coverpage
\section{Teknologia}

  \paraitem{Råsa} En lämplig tolkning av råsa är \texttt{0xF280A1}. Detta kan
  tolkas som Pantone 189 U eller 35,3\% svart.

  \paraitem{Rosa Pantern} Med Rosa Pantern åsyftas den figur som skapades av
  Friz Freleng och David DePatie till Blake Edwards film ``The Pink Panther'' år
  1963.

  \paraitem{Råsa Digitalis} Med Råsa Digitalis menas en råsa fingerborgsblomma,
  på latin Digitalis Purpurea.

\section{Hedersomnämnande}
  \paraitem{Hedersmedlemmar}
  \begin{itemize}
    \item Nina Reistad \\
      \emph{Inlagd i reglementet på VTM1 2000. Inröstad 1993}
    \item Per-Henrik Rasmussen \\ \emph{Inröstad 1994.}
    \item Rune Kullberg \\
      \emph{Invald på HTM1 2002 för sitt arbete för studentinflytande, hans
      handlingskraft i studentfrågor, och hans stora initiativförmåga som
      ordförande i Utbildningsnämnden för D}
    \item Nora Ekdahl \\
      \emph{Invald VTM 2010 för sina insatser som studievägledare för C- och
      D-programmet.}
    \item Mats Cedervall \\
      \emph{Invald på HTM1 2018 för sitt stora Intresse i sektionens lokaler och
      för att han som Husprefekt alltid har verkat för studenternas ökade
      inflytande i husstyrelsen}
    \item Jan Eric Larsson \\
      \emph{Invald på HTM1 2022 för sina insatser och sitt stora engagemang som
      Inspektor under en lång tid.}
    \item Anu Uus \\
      \emph{Invald på HTM1 2022 för sina insatser och sitt stora engagemang som
      Inspektrix under en lång tid.}
  \end{itemize}

  \paraitem{Grundarna av D-sektionen inom TLTH} Grundarna av D-sektionen inom
  TLTH är:
  \begin{itemize}
  \item Lars Svensson
  \item Ulf Bengtsson
  \item Henrik Cronström
  \item Kristin Andersson
  \item Stefan Molund
  \item Jan Eric Larsson
  \end{itemize}

\end{document}
