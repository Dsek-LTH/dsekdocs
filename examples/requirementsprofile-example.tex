\documentclass{dsekrequirementsprofile}
\usepackage{dsek}

\kravprofilför{Peppare} % Här sätts posten

\setauthor{Pepparvalberedningen} % Byt till relevant grupp
\setdate{\datum{1}{17} 20XX}     % Sätt datum. \today fungerar fint.

\begin{document}

\maketitle

\section*{Postbeskrivning}
Som Peppare jobbar du tillsammans med Peppargruppen under hela mandatperioden,
kalenderåret 2024. Som Peppare arbetar man med att planera och genomföra
roliga idéer och event innan samt under nollningen. Man arbetar också
tillsammans med Staben och Øverpepparna, exempelvis under Phaddervalet. En av
dina viktigaste uppgifter som Peppare är också att se till så att nollorna mår
bra, trivs och har kul!

\section*{Krav}
\begin{itemize}
\item Passionerad
\item Omtänksam
\item Respektfull
\item Samarbetsvillig
\item Kommunikationsförmåga
\item Handlingskraftig
\item Stresskännedom
\item Kompromissvillig
\item Självinsikt
\item Förståelse för postens innebörd
\item Intresse för eventplanering
\end{itemize}

\section*{Meriterande}
\begin{itemize}
\item Erfarenhet av att arrangera event
\item Peppig
\item Planeringsförmåga
\end{itemize}

\section*{Gruppkravprofil}      % Ta bort det här om det inte behövs
Gruppkravprofilen består av krav som gruppen som helhet behöver innehålla,
alltså måste du som individ inte uppfylla alla eller ens någon av kraven för att
bli vald. Vi väger in gruppkravprofilen när vi utför valet för att bilda den
bästa möjliga gruppen som helhet.

\begin{itemize}
\item Gillar att bjuda på sig själv
\item Idéspruta
\item Energisk
\item Erfarenhet inom att hålla event
\item Sektionskännedom
\end{itemize}

\end{document}
