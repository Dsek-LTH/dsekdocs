\documentclass{dsekstatutes}

\setadoptedon{HTMX 20XX}
\setrevisedon{VTMX 20XY}

\begin{document}
\coverpage
\section{Sektionen}
\paraitem{Namn}
Sektionens namn är D-sektionen inom Teknologkåren vid Lunds Tekniska Högskola,
hädanefter kallat sektionen. Kortare benämningar är D-sektionen inom TLTH,
D-sektionen och Dsek.

\paraitem{Ändamål}
Föreningens ändamål och syfte är att verka för sammanhållningen mellan sina
medlemmar, att främja deras studier och utbild- ning, att tillvarata deras
gemensamma intressen, samt vad därmed äger sammanhang. Sektionen drivs utan
vinstintresse.

\paraitem{Organisationsnummer}
Sektionens organisationsnummer är 845003-2878.

\paraitem{Symbol}
Sektionens officiella symbol ser ut enligt nedanstående bild.

\Dsymbol[height=10mm]

\paraitem{Sigill}
Sektionens officiella sigill ser ut enligt nedanstående bild.

\Dseksigil[height=30mm, color]

\paraitem{Färg}
Sektionens färg är Råsa.

\paraitem{Maskot}
Sektionens maskot är Rosa Pantern.

\paraitem{Blomma}
Sektionens blomma är Råsa Digitalis.

\paraitem{Hymn}
Sektionshymnen är ”Rosa på bal” av E. Taube.

\paraitem{Överklagande}
Beslut taget av någon av sektionens myndigheter kan hos kårens fullmäktige
överklagas av minst en tiondel, eller etthundra av sektionens medlemmar inom tre
veckor från den dag beslutet tillkännagetts.

\paraitem{Undanröjande}
Beslut taget av någon av sektionens myndigheter kan av kårens fullmäktige
undanröjas endast om det uppenbart strider mot ändamålsparagrafen §1.2 eller mot
ändamålsparagrafen i kårens stadgar.

\paraitem{Stadfästande}
Stadfästelse innebär att en högre instans bekräftar ett beslut taget av en
instans underordnad den högre instansen. Ett beslut som behöver stadfästas
gäller från den tidpunkt beslutet togs, men kan rivas upp, ändras eller
godkännas vid tiden för stadfästelse.

\end{document}
