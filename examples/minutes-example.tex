% Use this for a guild meeting,
% \documentclass[guild]{dsekminutes}

% use this for a board meeting,
% \documentclass[board]{dsekminutes}

% or this for a study-council meeting.
% \documentclass[srd]{dsekminutes}

% If you leave it blank, it will default to guild meeting
\documentclass{dsekminutes}

% Set information about the meeting
\setdate{\datum[2069]{03}{28}}
\settitle{Protokoll Vårterminsmöte 2069}
\setmeeting{VTM}
\setauthor{Fanny Adolvssson}

\begin{document}

\maketitle

% Take attendance
\section*{Närvaro}
\begin{attendance}
  \person{Alice Haraldsson}[Talman]
  \person{Britta Ottosson}[Ordförande]
  \person{Fanny Adolvsson}[Vice Ordförande]
  \person{Fanny Albrechtsson}[Skattmästare]
  \person{Håkan Carlsson}[Informationsansvarig]
  \person{Idun Strömberg}
  \person{Kjell Sörensson}
  \person{Liselott Alvarsson}
  \person{Pauline Ljung}
  \person{Selma Dahl}
  \person{Siv Hult}
  \person{Sixten Öberg}
  \person{Teodora Ericson}
  \person{Vanja Stigsson}
  \person{Vera Robertsson}
\end{attendance}

\section*{Beslutsprotokoll}

\paragraph{Talman förklarar mötet öppnat}
Talman Alice Haraldsson förklarade mötet öppnat klockan 17:40.

\paragraph{Tid och sätt}
Mötet \beslöt att godkänna tid och sätt.

\paragraph{Val av justeringspersoner tillika rösträknare}
Liselott Alvarsson kandiderade.  Siv Hult kandiderade.

Mötet \beslöt att välja
\begin{itemize}
\item Liselott Alvarsson,
\item Siv Hult
\end{itemize}
till justeringspersoner tillika rösträknare.

\paragraph{Föredragningslista}
Britta Ottosson \yrkade på att lägga till ``Exempelpunkt''
som ny §7.

Mötet \beslöt
\begin{attlist}
  \att bifalla Britta Ottossons yrkande,
  \att godkänna den framvaskade föredragningslistan.
\end{attlist}

\paragraph{Föregående mötesprotokoll}
Pauline Ljung \yrkade på
\begin{attlist}
  \att lägga sektionsmötesprotokollet för HTM2 till handlingarna,
  \att lägga sektionsmötesprotokollet för HTM-val till handlingarna.
\end{attlist}

Mötet \beslöt
\begin{attlist}
  \att bifalla Pauline Ljungs första yrkande,
  \att bifalla Pauline Ljungs andra yrkande.
\end{attlist}

\paragraph{Verksamhetsrapport}
Styrelsen presenterade verksamhetsrapporten.

Styrelsen utfrågades.

Mötet \beslöt
\begin{attlist}
  \att lägga “Verksamhetsrapporten” till handlingarna.
\end{attlist}

\paragraph{Exempelpunkt}
Britta Ottosson föredrog punkten.

Britta Ottosson \yrkade på att lägga punkten till handlingarna.

\paragraph{Motion: Inköp av espressomaskin}
Sixten Öberg föredrog motionen.

Håkan Carlsson \ändringsyrkade på
\begin{attlist}
  \att ändra den första attsatsen till \emph{sektionen köper in en
  espresso\-maskin för en kostnad som uppgår till max 50000 \times 1,15 =
  57500kr.}
\end{attlist}

Mötet \beslöt
\begin{attlist}
  \att bifalla Håkan Carlssons yrkande,
  \att bifalla den framvaskade motionen.
\end{attlist}

\paragraph{Talman förklarar mötet avslutat}
Alice Haraldsson förklarade mötet avslutat 19:36.

\vspace{5cm}

\signature{Vid protokollet}{Fanny Adolvsson}{Vice Ordförande}
\signature{Mötesordförande}{Alice Haraldsson}{Talman}
\signature{Justeras}{Liselott Alvarsson}{Justeringsperson}
\signature{Justeras}{Siv Hult}{Justeringsperson}

\end{document}
